\section{Линейные представления (действия) групп}

\subsection{Определения}

Зафиксируем поле $\K$, над которым будем рассматривать векторное пространство
$V(+, \cdot, \cdot)$ (умножение на скаляр и на элементы $G$).

\begin{df}
	Задано \textit{линейное действие}, если задано умножение элементов из $V$
	слева на элементы из $G$, $\fA g \in G, v \in V (g, v) \mapsto gv \in V$,
	\sth\ $\fA v, v_1, v_2 \in V \fA g, h \in G, \fA \la \in \K$
	\begin{points}{-3}
		\item $(gh)v = g(hv)$
		\item $ev = v$
		\item $g(v_1 + v_2) = gv_1 + gv_2$
		\item $g(\la v) = \la (gv)$
	\end{points}

	\textit{Линейное представление} $\rh \colon G \to \GL(V)$, $\rh(g)(v) = gv$
	и обратно $gv = \rh(g)(v)$.
\end{df}
\begin{denote}
	$(G, V, \rh)$ эквивалентно записи $\rh = (G, V)$.
\end{denote}
\begin{df}
	Подпространство $U \subseteq V$ является \textit{подпредставлением}, если
	оно инвариантно относительно действий элементов $G$, \ie\ 
	$\fA u \in U \fA g \in G$ $gu \in U$. 
\end{df}
\begin{df}
	Пусть $V$ --- представление, и его инвариантное подпространство $U$.
	Тогда \textit{факторпредставление} $V / U = \hc{v + U | v \in V}$.
	Зададим операцию $g(v + U) = gv + U$.
	Проверим корректность.
	Возьмём два разных предстваителя: $v_1 + U = v_2 + U$, \ie\ $v_1 - v_2 \in U$.
	Достаточно, что бы $gv_1 + U = gv_2 + U$.
	Но, так как $U$ инвариантно, $g(v_1 - v_2) \in U$.
\end{df}

\subsection{Прямая сумма представлений}

\begin{df}
	Пусть заданы инвариантные подпространства $U_1\sco U_s \subset V$,
	$V = U_1 \sop U_s$ --- разложение в \textit{(внутреннею) прямую сумму} инвариантных подпространств (подпредставлений).
	$\rho_1 = (G, U_1)$, $\rho_2 = (G, U_2)$, \ldots,$\rho_s = (G, U_s)$ 
\end{df}

\begin{df}
	\textit{Внешняя прямая сумма представлений} $V = V_1 \sop V_s = \hc{\hr{v_1 \sco v_s} | v_i \in V_i}$,
	$g(v_1 \sco v_s) = (gv_1 \sco gv_s)$, далее будем \textbf{обозначать}
	$\rho = \rho_1 \sop \rho_s$.
\end{df}

Пусть задан гомоморфизм $H \xra{f} G \xra{\rho} \GL(V)$.
Тогда композиция $f$ и $\rho$ даст представление $H: hv = f(h)v$


\subsection{Гомоморфизмы представлений}

Пусть имеем два представления: $\rho_1 = (G, V_1)$, $\rho_2 = (G, V_2)$. 
\begin{df}
	\textit{Гомоморфизм представлений} $\phi \cln \rho_1(V_1) \to \rho_2(V_2)$
	есть линейное отображение \sth\ 
	$\phi(gv) = g\phi(v), \fA g \in G, v \in V$, \ie\ 
	$\fA g \in G$ коммутативна диаграма
	$$
		\xymatrix{
				V_1 \ar[r]^{\phi} \ar[d]_{\rho_1(g)} & V_2 \ar[d]^{\rho_2(g)} \\
				V_1 \ar[r]^{\phi} & V_2
			}
	$$
\end{df}

\begin{df}
	\textit{Изоморфизм представлений} --- это гомоморфизм, который является биекцией.
\end{df}

\subsection{Матричные представления}

$\rho = (G, V)$, $\rho \cln G \to \GL(V)$.
\textbf{Всегда} будем считать, что $V$ --- конечномерное пространство.
$V = \ha{e_1 \sco e_n}$ $\Ra$ $\GL(V) \cong \GL(n, \K)$.
Рассмотрим сопоставление $\rho(g) \mapsto A_g$ --- матрица $\rho(g)$ относительно $e_1 \ sco e_n$.
\begin{df}
	Гомоморфизм $G \to \GL(n, \K)$ будем называть матричным представлением.
	Так же потребуем:
 	\begin{points}{-3}
 		\item $A_{gh} = A_g A_h$
 		\item $A_e = E$
 		\item $A_{g^{-1}} = {(A_g)}^{-1}$
 	\end{points}
\end{df}

Если задано матричное представление,
то можем построить линейный оператор $\Ra$
матричное и линейное представления равносильны
(хотя в одном случае неоднозначно).

Пусть $V = \ha{e_1 \sco e_n} = \ha{e'_1 \sco e'_n}$ и $C$ --- матрица перехода.
\begin{align*}
	g &\mapsto A_g & \hr{e_1 \sco e_n} \\
	g &\mapsto A'_g & \hr{e_1 \sco e_n} \\
	A'_g & = C^{-1} A_g C & \fA g \in G
\end{align*}

\begin{df}
	Два матричных представления называются \textit{эквивалентными},
	если 
		$\Ex C \cln A'_g = C^-1 A_g C \fA g \in G$.
\end{df}

\begin{stm}
	Два линейных представления изоморфны $\Lra$
	соответсвущие матричные представления относительно некоторых базисов эквивалентны.
\end{stm}
\begin{proof}
	$\Ra$: Имеем представление $\rho = (G, V)$.
	Пусть имеется $\rho' = (G, V')$ и $\fA g \in G$ коммутативна
	$$
		\xymatrix{
				V_1 \ar[r]^{\phi} \ar[d]_{\rho_1(g)} & V' \ar[d]^{\rho'(g)} \\
				V_1 \ar[r]^{\phi} & V'
			}
	$$
	Выберем базисы в пространствах $V$ и $V'$,
	$V = \ha{e_1 \sco e_n}$, $V' = \ha{e'_1 \sco e'_n}$, $\rho = \hc{A_g}$, $\rho' = \hc{A'_g}$.

	Пусть $C$ --- матрица для $\phi$ относительно выбранных базисов.
	\Bt\ изоморфизм, то $\det C \ne 0$.

	Композиции линейных отображений соответсвует матрица $\Ra$
	$A'_g C = C A_g$ $\Ra$ $A'_g = C A_g C^{-1}$ $\Ra$ эквиваленты.

	$\Lar$: Пусть матричные представления эквивалентны относительно некотрых базисов $\Ra$
	$\Ex C \cln A'_g = C A_g C^{-1}$

	Но матрица $C$ относительно базисов $\ha{e_1 \sco e_n}$ и $\ha{e'_1 \sco e'_n}$ $\Ra$
	невырождено отображение $A'_g C = C A_g$ $\Ra$ $\rho'(g) \circ \phi = \phi \circ \rho(g)$ $\Ra$
	линейные представления изоморфны.
\end{proof}


\subsection{Приводимые, неприводимые и вполне приводимые линейные представления}

\begin{df}
	Представление $\rho$ --- приводимое, если оно имеет
	подпредставление на инвариантном подпространстве,
	отличном от тривиальных
\end{df}
\begin{df}
	Представление $\rho$ --- неприводимое, если не существует
	инвариантных подпространств отличных от тривиальных.
\end{df}
\begin{df}
	Представление вполне приводимо, если оно разлагается в прямую сумму неприводимых.
\end{df}

На матричном языке:

Пусть $\rho$ приводимо $\Ra$
$0 \ne U \subsetneq V$ --- инвариантное подпространство.
Выберем базис так: $V = \ha{e_1 \sco e_k, e_{k + 1} \sco e_n}$,
$$
	\rho(g) = \rbmat{B_g & * \\ 0 & C_g} = A_g,
$$
где $\rho(g)(e_i) \in U$, $i = 1 \sco k$;
$\hc{B_g}$ соответсвует $\rho\evn{U}$.

На $V/U$ также имеется индуцированное представление:
$V/U = \ha{e_{k+1} + U \sco e_n + U}$, $g(v + U) = gv + U$.
Тогда $g(e_i + U) = g e_i + U$ достаточно задать на базисных векторах.

Если базис выбран произвольным образом, $C \cln \det C \ne 0$,
то $\hc{C^{-1} A'_g C}$ будут иметь общий угол нулей ($C$ одна для всех $g$).

Пусть $\rho = \rho_1 \sop \rho_s$, $\rho_i = (G, V_i)$, $V = V_1 \sop V_s$,
$V_i$ --- инвариантные подпространства в $V$.

Выберем базис в $V_i$ и в качестве базиса $V$ берём объединение базисов $V_i$.
Тогда
$$
	\rho(g) = A_g = \rbmat{A^{(1)}_g && 0 \\ & \ddots & \\ 0 && A^{(s)}_g}
$$
есть прямая сумма диагональных блоков.

Вполне приводимое, если каждая матрица ---
прямая сумма неприводимых блоков (в блоке нет угла нулей) $\Ra$
при любом выборе базиса будем получать матрицы, эквивалентные неприводимым.

\subsection{Конечномерное представление циклической группы над $\Cx$}
Пусть $G = \ha{a}$. Рассмотрим $\rho \cln G \to \GL(n, \Cx)$
\begin{points}{-3}
\item $G = {\ha{a}}_{\infty}$. Достаточно задать $\rho(a)$.
	Положим $\rho(a) = A \in \GL(n, \Cx)$ --- любая матрица.
	$\rho' \sim \rho$ $\Ra$ $\Ex C \cln A' = C^{-1} A C$ $\Ra$
	если верно для $A$, $\hm{C} \ne 0$, то верно и для сопряженной.

	\begin{theorem}[из линала]
		Матрицы сопряженны $\Lra$ сопряженны их жордановы формы
	\end{theorem}
	Тогда матрица $\rho(a)$ задаётся жордановой формой $\Ra$
	размеры клеток определены однозначно.
	$$
		C A C^{-1} = \rbmat{
					\mat{\lambda'_1 && \\ & \ddots & \\ && \lambda'_1} && 0 \\
					& \ddots & \\
					0 && \mat{\lambda'_s && \\ & \ddots & \\ && \lambda'_s}}
	$$
	Если есть жорданова клетка размерности $\ge 2$, то представление не вполне приводимо.
	Значит, вполне приводимо $\Lra$ матрица $A$ диаганализуема.
\item $G = {\ha{a}}_n$, $\rho(a) = A$, $a^n = e$ $\Ra$ $A^n = E$.
	Тогда $t^n - 1$ --- аннулирующий для $A$.
	Но над $\Cx$ этот многочлен не имеет кратных корней $\Ra$
	матрица диаганализуема:
	\begin{gather*}
		{\lambda_i}^n = 1 \\
		C^{-1} A C = \rbmat{\lambda_1 && 0 \\ & \ddots & \\ 0 && \lambda_n}
	\end{gather*}
	$\Ra$ Любое представление конечной циклической группы (вполне) приводимо.
	Матрицы не эквивалетны $\Lra$ имеют разные характеристические многочлены.
\end{points}


\subsection{Неприводимые представления абелевых групп над $\Cx$}
\begin{theorem}
	Над $\Cx$ представление абелевой группы неприводимо $\Lra$
	оно одномерное.
\end{theorem}
\begin{proof}
	\begin{theorem}
		Пусть $V$ --- конечномерное пространство, $\dim V = n$.
		$\hc{\phi_i}$ --- некоторое семейство попарно коммутирующих линенйных
		операторов на $V$ над $\Cx$.
		Тогда они имеют общий собственный вектор.
	\end{theorem}
	\begin{proof}
		Индукция по размерности $n$:
		\begin{points}{-3}
			\item $n = 1$ --- все собственные
			\item Пусть $n > 1$. Если все $\phi_i = \lambda_i \epsilon$, то доказывать нечего.
				Пусть $\phi_1$ не скалярный $\Ra$
				$\phi_1$ имеет собственный вектор, \ie\
				$\phi_1(e) = \lambda e$, $\lambda \in \Cx$

				Рассмотрим подпространство $V_{\lambda}$ всех собственных векторов
					со значением $\lambda$.
				$0 \ne V_{\lambda} = \Ker (\phi_1 - \lambda \epsilon) \subsetneq V$, \bt\
				не $\phi_1$ не скалярный $\Ra$ $1 \le \dim V_{\lambda} < n$.

				Покажем, что $V_{\lambda}$ --- инвариатное подпространство, через перестановочность операторов.

				Пусть $v \in V_{\lambda}$, $\phi_i(v) \in V_{\lambda}$ $\Lra$
				$\phi_1(\phi_i(v)) = \lambda \phi_i(v)$.
				Но $\phi_1 \phi_i = \phi_i \phi_1$ $\Ra$
				$$
					\phi_1(\phi_i(v)) = \phi_i(\phi_1(v)) = 
					\phi_i(\lambda v ) = \lambda \phi_i(v).
				$$

				Рассмотрим $\hc{\phi_i\evn{V_{\lambda}}}$, $\dim V_{\lambda} < n$ $\Ra$
				можем применить индуктивное предположение $\Ra$
				$\phi_i$ имеют общий собственный вектор в $V_{\lambda}$ $\Ra$ и в $V$.
		\end{points}
	\end{proof}

	Пусть $G$ --- абелева, $\rho$ --- неприводимое над $\Cx$.
	$\hc{\rho(g) | g \in G}$ --- семейтсво попарно коммутирующих операторов (\bt\ абелева группа) $\Ra$
	%TODO: good reference
	по теореме (1.4) $\Ex 0 \ne v \in V \cln \rho(g)(v) = \lambda_g v$,
	но тогда $V \supset \ha{v}$ --- инвариантное подпространство в $V$ $\Ra$ $V = \ha{v}$
\end{proof}

Пусть имеем произвольное поле $\K$, $\rho = (G, V)$, $\dim V = 1$.
$\rho \cln G \to \GL(1, \K) = \K^*$.
Тогда для $\rho' \cln G \to \K^*$
$\hc{a_g}$, $\hc{a'_g}$ $\Ex C \in \K^* \cln a_g = C^{-1} a_g C = a_g$ $\Ra$
в одномерном случае эквивалентность --- совпадение гоморфизмов $\Ra$
надо найти все гомоморфизмы $G \to \K^*$.

$\hm{G} = n$ --- абелева группа, $\K = \Cx$.
Найдём все комплексные представления конечной абелевой группы
$$
	G = {\ha{a_1}}_{n_1} \sop {\ha{a_s}}_{n_s} \xra{\rho} \Cx^*
$$
Достаточно задать $\rho$ на $a_i$, но $a_i^{n_i} = e$ $\Ra$
$\hr{\rho(a_i)}^{n_i} = 1$ $\Ra$
$\rho(a_i) = \xi_i \in \sqrt[n_i]1$ $\Ra$
имеем гомоморфизм каждого слагаемого в $\Cx^*$.

$G = \ha{a_1} \st \ha{a_s}$, $\rho(a_1^{k_1} \sd a_s^{k_s}) = \xi_1^{k_1} \sd \xi_s^{k_s}$,
$k_i \in \Z$, $k_i = 0 \sco n_i - 1$.
Проверим, что $\rho$ --- гомоморфизм прямого произведения:
\begin{gather*}
	\rho((a_1^{k_1} \sd a_s^{k_s})(a_1^{l_1} \sd a_s^{l_s})) = 
	\rho((a_1^{k_1} a_1^{l_1}) \sd (a_s^{k_s} a_s^{l_s})) = \\
	= (\xi_1^{k_1} \xi_1^{l_1}) \sd (\xi_s^{k_s} \xi_s^{l_s}) = 
	(\xi_1^{k_1} \sd \xi_s^{k_s}) (\xi_1^{l_1} \sd \xi_s^{l_s}) = 
	\rho(a_1^{k_1} \sd a_s^{k_s}) \rho(a_1^{l_1} \sd a_s^{l_s})
\end{gather*}
\begin{stm}
	Если имеется гомоморфизм произведения в абелеву группу,
	то возможностей выбрать $\xi_i$\clue{}-ые $n_1 \sd n_s = n$
\end{stm}
\begin{comm}
	Доказывалось ранее в более общем виде.
\end{comm}
Так, число различных одномерных $\Cx$\clue{}-представлений абелевой группы равно её порядку.


\subsection{Одномерные представления конечной группы}
$\rho \cln G \to \K^*$.
$\K^*$ --- коммутативна $\Ra$
$\Im \rho \cong G / \Ker \rho$ --- абелева.
Факторгруппа абелева $\Lra$ $G' \subseteq \Ker \rho$
$\Ra$ нужны только такие гомоморфизмы.

Пусть $N \lhd G$, $\rho \cln G \to H$, $N \subseteq \Ker \rho$.
Такие гомоморфизмы находятся в биективном соответсвии с гомоморфизмами $G/N \to H$.

Одномерные представления $G$ над $\K$ находятся в биективном соответсвии с гомоморфизмами
$G / G' \xra{\ol{\rho}} K^*$, $\rho = \ol{\rho} \circ \pi$ $\Ra$
задача сводится к представлению абелевой группы.

Пусть $\K^* = \Cx^*$, $\hm{G} = n < \infty$ $\Ra$ $G/G'$ --- конечная абелева группа.
$\hm{G/G'}$ разных гомоморфизмов абелевого фактора $\Ra$
число одномерных представлений конечной группы $G$ есть порядок $G/G'$


\subsection{Пространсва гомоморфизмов линейных представлений групп}
%TODO:52
