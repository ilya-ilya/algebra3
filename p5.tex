\section{Линейные представления (действия) групп}

\subsection{Определения}

Зафиксируем поле $\K$, над которым будем рассматривать векторное пространство
$V(+, \cdot, \cdot)$ (умножение на скаляр и на элементы $G$).

\begin{df}
	Задано \textit{линейное действие}, если задано умножение элементов из $V$
	слева на элементы из $G$, $\fA g \in G, v \in V (g, v) \mapsto gv \in V$,
	\sth\ $\fA v, v_1, v_2 \in V \fA g, h \in G, \fA \la \in \K$
	\begin{points}{-3}
		\item $(gh)v = g(hv)$
		\item $ev = v$
		\item $g(v_1 + v_2) = gv_1 + gv_2$
		\item $g(\la v) = \la (gv)$
	\end{points}

	\textit{Линейное представление} $\rh \colon G \to \GL(V)$, $\rh(g)(v) = gv$
	и обратно $gv = \rh(g)(v)$.
\end{df}
\begin{denote}
	$(G, V, \rh)$ эквивалентно записи $\rh = (G, V)$.
\end{denote}
\begin{df}
	Подпространство $U \subseq V$ является \textit{подпредставлением}, если
	оно инвариантно относительно действий элементов $G$, \ie\ 
	$\fA u \in U \fA g \in G$ $gu \in U$. 
\end{df}
\begin{df}
	Пусть $V$ --- представление, и его инвариантное подпространство $U$.
	Тогда \textit{факторпредставление} $V / U = \hc{v + U | v \in V}$.
	Зададим операцию $g(v + U) = gv + U$.
	Проверим корректность.
	Возьмём два разных предстваителя: $v_1 + U = v_2 + U$, \ie\ $v_1 - v_2 \in U$.
	Достаточно, что бы $gv_1 + U = gv_2 + U$.
	Но, так как $U$ инвариантно, $g(v_1 - v_2) \in U$.
\end{df}

\subsection{Прямая сумма представлений}

\begin{df}
	Пусть заданы инвариантные подпространства $U_1\sco U_s \subs V$,
	$V = U_1 \sop U_s$ --- разложение в \textit{(внутреннею) прямую сумму} инвариантных подпространств (подпредставлений).
	$\rho_1 = (G, U_1)$, $\rho_2 = (G, U_2)$, \ldots,$\rho_s = (G, U_s)$ 
\end{df}

\begin{df}
	\textit{Внешняя прямая сумма представлений} $V = V_1 \sop V_s = \hc{\hr{v_1 \sco v_s} | v_i \in V_i}$,
	$g(v_1 \sco v_s) = (gv_1 \sco gv_s)$, далее будем \textbf{обозначать}
	$\rho = \rho_1 \sop \rho_s$.
\end{df}

Пусть задан гомоморфизм $H \xra{f} G \xra{\rho} \GL(V)$.
Тогда композиция $f$ и $\rho$ даст представление $H: hv = f(h)v$


\subsection{Гомоморфизмы представлений}

Пусть имеем два представления: $\rho_1 = (G, V_1)$, $\rho_2 = (G, V_2)$. 
\begin{df}
	\textit{Гомоморфизм представлений} $\phi \cln \rho_1(V_1) \to \rho_2(V_2)$
	есть линейное отображение \sth\ 
	$\phi(gv) = g\phi(v), \fA g \in G, v \in V$, \ie\ 
	$\fA g \in G$ коммутативна диаграма
	$$
		\xymatrix{
				V_1 \ar[r]^{\phi} \ar[d]_{\rho_1(g)} & V_2 \ar[d]^{\rho_2(g)} \\
				V_1 \ar[r]^{\phi} & V_2
			}
	$$
\end{df}

\begin{df}
	\textit{Изоморфизм представлений} --- это гомоморфизм, который является биекцией.
\end{df}

\subsection{Матричные представления}

$\rho = (G, V)$, $\rho \cln G \to \GL(V)$.
\textbf{Всегда} будем считать, что $V$ --- конечномерное пространство.
$V = \ha{e_1 \sco e_n}$ $\Ra$ $\GL(V) \cong \GL(n, \K)$.
Рассмотрим сопоставление $\rho(g) \mapsto A_g$ --- матрица $\rho(g)$ относительно $e_1 \ sco e_n$.
\begin{df}
	Гомоморфизм $G \to \GL(n, \K)$ будем называть \textit{матричным представлением}.
	Так же потребуем:
 	\begin{points}{-3}
 		\item $A_{gh} = A_g A_h$
 		\item $A_e = E$
 		\item $A_{g^{-1}} = {(A_g)}^{-1}$
 	\end{points}
\end{df}

Если задано матричное представление,
то можем построить линейный оператор $\Ra$
матричное и линейное представления равносильны
(хотя в одном случае неоднозначно).

Пусть $V = \ha{e_1 \sco e_n} = \ha{e'_1 \sco e'_n}$ и $C$ --- матрица перехода.
\begin{align*}
	g &\mapsto A_g & \hr{e_1 \sco e_n} \\
	g &\mapsto A'_g & \hr{e_1 \sco e_n} \\
	A'_g & = C^{-1} A_g C & \fA g \in G
\end{align*}

\begin{df}
	Два матричных представления называются \textit{эквивалентными},
	если 
		$\Ex C \cln A'_g = C^-1 A_g C \fA g \in G$.
\end{df}

\begin{stm}
	Два линейных представления изоморфны $\Lra$
	соответсвущие матричные представления относительно некоторых базисов эквивалентны.
\end{stm}
\begin{proof}
	$\Ra$: Имеем представление $\rho = (G, V)$.
	Пусть имеется $\rho' = (G, V')$ и $\fA g \in G$ коммутативна
	$$
		\xymatrix{
				V_1 \ar[r]^{\phi} \ar[d]_{\rho_1(g)} & V' \ar[d]^{\rho'(g)} \\
				V_1 \ar[r]^{\phi} & V'
			}
	$$
	Выберем базисы в пространствах $V$ и $V'$,
	$V = \ha{e_1 \sco e_n}$, $V' = \ha{e'_1 \sco e'_n}$, $\rho = \hc{A_g}$, $\rho' = \hc{A'_g}$.

	Пусть $C$ --- матрица для $\phi$ относительно выбранных базисов.
	\Bt\ изоморфизм, то $\det C \ne 0$.

	Композиции линейных отображений соответсвует матрица $\Ra$
	$A'_g C = C A_g$ $\Ra$ $A'_g = C A_g C^{-1}$ $\Ra$ эквиваленты.

	$\Lar$: Пусть матричные представления эквивалентны относительно некотрых базисов $\Ra$
	$\Ex C \cln A'_g = C A_g C^{-1}$

	Но матрица $C$ относительно базисов $\ha{e_1 \sco e_n}$ и $\ha{e'_1 \sco e'_n}$ $\Ra$
	невырождено отображение $A'_g C = C A_g$ $\Ra$ $\rho'(g) \circ \phi = \phi \circ \rho(g)$ $\Ra$
	линейные представления изоморфны.
\end{proof}


\subsection{Приводимые, неприводимые и вполне приводимые линейные представления}

\begin{df}
	Представление $\rho$ --- \textit{приводимое}, если оно имеет
	подпредставление на инвариантном подпространстве,
	отличном от тривиальных
\end{df}
\begin{df}
	Представление $\rho$ --- \textit{неприводимое}, если не существует
	инвариантных подпространств отличных от тривиальных.
\end{df}
\begin{df}
	Представление \textit{вполне приводимо}, если оно разлагается в прямую сумму неприводимых.
\end{df}

На матричном языке:

Пусть $\rho$ приводимо $\Ra$
$0 \ne U \subsneq V$ --- инвариантное подпространство.
Выберем базис так: $V = \ha{e_1 \sco e_k, e_{k + 1} \sco e_n}$,
$$
	\rho(g) = \rbmat{B_g & * \\ 0 & C_g} = A_g,
$$
где $\rho(g)(e_i) \in U$, $i = 1 \sco k$;
$\hc{B_g}$ соответсвует $\rho\evn{U}$.

На $V/U$ также имеется индуцированное представление:
$V/U = \ha{e_{k+1} + U \sco e_n + U}$, $g(v + U) = gv + U$.
Тогда $g(e_i + U) = g e_i + U$ достаточно задать на базисных векторах.

Если базис выбран произвольным образом, $C \cln \det C \ne 0$,
то $\hc{C^{-1} A'_g C}$ будут иметь общий угол нулей ($C$ одна для всех $g$).

Пусть $\rho = \rho_1 \sop \rho_s$, $\rho_i = (G, V_i)$, $V = V_1 \sop V_s$,
$V_i$ --- инвариантные подпространства в $V$.

Выберем базис в $V_i$ и в качестве базиса $V$ берём объединение базисов $V_i$.
Тогда
$$
	\rho(g) = A_g = \rbmat{A^{(1)}_g && 0 \\ & \ddots & \\ 0 && A^{(s)}_g}
$$
есть прямая сумма диагональных блоков.

Вполне приводимое, если каждая матрица ---
прямая сумма неприводимых блоков (в блоке нет угла нулей) $\Ra$
при любом выборе базиса будем получать матрицы, эквивалентные неприводимым.

\subsection{Конечномерное представление циклической группы над $\Cx$}
Пусть $G = \ha{a}$. Рассмотрим $\rho \cln G \to \GL(n, \Cx)$
\begin{points}{-3}
\item $G = {\ha{a}}_{\infty}$. Достаточно задать $\rho(a)$.
	Положим $\rho(a) = A \in \GL(n, \Cx)$ --- любая матрица.
	$\rho' \sim \rho$ $\Ra$ $\Ex C \cln A' = C^{-1} A C$ $\Ra$
	если верно для $A$, $\hm{C} \ne 0$, то верно и для сопряженной.

	\begin{theorem}[из линала]
		Матрицы сопряженны $\Lra$ сопряженны их жордановы формы
	\end{theorem}
	Тогда матрица $\rho(a)$ задаётся жордановой формой $\Ra$
	размеры клеток определены однозначно.
	$$
		C A C^{-1} = \rbmat{
					\mat{\lambda'_1 && \\ & \ddots & \\ && \lambda'_1} && 0 \\
					& \ddots & \\
					0 && \mat{\lambda'_s && \\ & \ddots & \\ && \lambda'_s}}
	$$
	Если есть жорданова клетка размерности $\ge 2$, то представление не вполне приводимо.
	Значит, вполне приводимо $\Lra$ матрица $A$ диаганализуема.
\item $G = {\ha{a}}_n$, $\rho(a) = A$, $a^n = e$ $\Ra$ $A^n = E$.
	Тогда $t^n - 1$ --- аннулирующий для $A$.
	Но над $\Cx$ этот многочлен не имеет кратных корней $\Ra$
	матрица диаганализуема:
	\begin{gather*}
		{\lambda_i}^n = 1 \\
		C^{-1} A C = \rbmat{\lambda_1 && 0 \\ & \ddots & \\ 0 && \lambda_n}
	\end{gather*}
	$\Ra$ Любое представление конечной циклической группы (вполне) приводимо.
	Матрицы не эквивалетны $\Lra$ имеют разные характеристические многочлены.
\end{points}


\subsection{Неприводимые представления абелевых групп над $\Cx$}
\begin{theorem}
	Над $\Cx$ представление абелевой группы неприводимо $\Lra$
	оно одномерное.
\end{theorem}
\begin{proof}
	\begin{theorem}
		Пусть $V$ --- конечномерное пространство, $\dim V = n$.
		$\hc{\phi_i}$ --- некоторое семейство попарно коммутирующих линенйных
		операторов на $V$ над $\Cx$.
		Тогда они имеют общий собственный вектор.
	\end{theorem}
	\begin{proof}
		Индукция по размерности $n$:
		\begin{points}{-3}
			\item $n = 1$ --- все собственные
			\item Пусть $n > 1$. Если все $\phi_i = \lambda_i \epsilon$, то доказывать нечего.
				Пусть $\phi_1$ не скалярный $\Ra$
				$\phi_1$ имеет собственный вектор, \ie\
				$\phi_1(e) = \lambda e$, $\lambda \in \Cx$

				Рассмотрим подпространство $V_{\lambda}$ всех собственных векторов
					со значением $\lambda$.
				$0 \ne V_{\lambda} = \Ker (\phi_1 - \lambda \epsilon) \subsneq V$, \bt\
				не $\phi_1$ не скалярный $\Ra$ $1 \le \dim V_{\lambda} < n$.

				Покажем, что $V_{\lambda}$ --- инвариатное подпространство, через перестановочность операторов.

				Пусть $v \in V_{\lambda}$, $\phi_i(v) \in V_{\lambda}$ $\Lra$
				$\phi_1(\phi_i(v)) = \lambda \phi_i(v)$.
				Но $\phi_1 \phi_i = \phi_i \phi_1$ $\Ra$
				$$
					\phi_1(\phi_i(v)) = \phi_i(\phi_1(v)) = 
					\phi_i(\lambda v ) = \lambda \phi_i(v).
				$$

				Рассмотрим $\hc{\phi_i\evn{V_{\lambda}}}$, $\dim V_{\lambda} < n$ $\Ra$
				можем применить индуктивное предположение $\Ra$
				$\phi_i$ имеют общий собственный вектор в $V_{\lambda}$ $\Ra$ и в $V$.
		\end{points}
	\end{proof}

	Пусть $G$ --- абелева, $\rho$ --- неприводимое над $\Cx$.
	$\hc{\rho(g) | g \in G}$ --- семейтсво попарно коммутирующих операторов (\bt\ абелева группа) $\Ra$
	%TODO: good reference
	по теореме (1.4) $\Ex 0 \ne v \in V \cln \rho(g)(v) = \lambda_g v$,
	но тогда $V \sups \ha{v}$ --- инвариантное подпространство в $V$ $\Ra$ $V = \ha{v}$
\end{proof}

Пусть имеем произвольное поле $\K$, $\rho = (G, V)$, $\dim V = 1$.
$\rho \cln G \to \GL(1, \K) = \K^*$.
Тогда для $\rho' \cln G \to \K^*$
$\hc{a_g}$, $\hc{a'_g}$ $\Ex C \in \K^* \cln a_g = C^{-1} a_g C = a_g$ $\Ra$
в одномерном случае эквивалентность --- совпадение гоморфизмов $\Ra$
надо найти все гомоморфизмы $G \to \K^*$.

$\hm{G} = n$ --- абелева группа, $\K = \Cx$.
Найдём все комплексные представления конечной абелевой группы
$$
	G = {\ha{a_1}}_{n_1} \sop {\ha{a_s}}_{n_s} \xra{\rho} \Cx^*
$$
Достаточно задать $\rho$ на $a_i$, но $a_i^{n_i} = e$ $\Ra$
$\hr{\rho(a_i)}^{n_i} = 1$ $\Ra$
$\rho(a_i) = \xi_i \in \sqrt[n_i]1$ $\Ra$
имеем гомоморфизм каждого слагаемого в $\Cx^*$.

$G = \ha{a_1} \st \ha{a_s}$, $\rho(a_1^{k_1} \sd a_s^{k_s}) = \xi_1^{k_1} \sd \xi_s^{k_s}$,
$k_i \in \Z$, $k_i = 0 \sco n_i - 1$.
Проверим, что $\rho$ --- гомоморфизм прямого произведения:
\begin{gather*}
	\rho((a_1^{k_1} \sd a_s^{k_s})(a_1^{l_1} \sd a_s^{l_s})) = 
	\rho((a_1^{k_1} a_1^{l_1}) \sd (a_s^{k_s} a_s^{l_s})) = \\
	= (\xi_1^{k_1} \xi_1^{l_1}) \sd (\xi_s^{k_s} \xi_s^{l_s}) = 
	(\xi_1^{k_1} \sd \xi_s^{k_s}) (\xi_1^{l_1} \sd \xi_s^{l_s}) = 
	\rho(a_1^{k_1} \sd a_s^{k_s}) \rho(a_1^{l_1} \sd a_s^{l_s})
\end{gather*}
\begin{stm}
	Если имеется гомоморфизм произведения в абелеву группу,
	то возможностей выбрать $\xi_i$\clue{}-ые $n_1 \sd n_s = n$
\end{stm}
\begin{comm}
	Доказывалось ранее в более общем виде.
\end{comm}
Так, число различных одномерных $\Cx$\clue{}-представлений абелевой группы равно её порядку.


\subsection{Одномерные представления конечной группы}
$\rho \cln G \to \K^*$.
$\K^*$ --- коммутативна $\Ra$
$\Im \rho \cong G / \Ker \rho$ --- абелева.
Факторгруппа абелева $\Lra$ $G' \subseq \Ker \rho$
$\Ra$ нужны только такие гомоморфизмы.

Пусть $N \lhd G$, $\rho \cln G \to H$, $N \subseq \Ker \rho$.
Такие гомоморфизмы находятся в биективном соответсвии с гомоморфизмами $G/N \to H$.

Одномерные представления $G$ над $\K$ находятся в биективном соответсвии с гомоморфизмами
$G / G' \xra{\ol{\rho}} K^*$, $\rho = \ol{\rho} \circ \pi$ $\Ra$
задача сводится к представлению абелевой группы.

Пусть $\K^* = \Cx^*$, $\hm{G} = n < \infty$ $\Ra$ $G/G'$ --- конечная абелева группа.
$\hm{G/G'}$ разных гомоморфизмов абелевого фактора $\Ra$
число одномерных представлений конечной группы $G$ есть порядок $G/G'$


\subsection{Пространсва гомоморфизмов линейных представлений групп}
Пусть $\K$ --- любое поле, $\rho = \hr{G, V}$, $\rho' = \hr{G, V'}$.
$\phi \in \Hom{\hr{\rho, \rho'}}$ --- множество гомоморфизмов $\rho \to \rho'$.
$V \xra{\phi} V'$ --- линейное отображение \sth\
$\fA v \in V, g \in G \phantom{\cln} \phi(gv) = g\phi(v)$.
$\Hom{\hr{\rho, \rho'}} = \Hom_G{\hr{V, V'}} \subseq \Lb\hr{V, V'}$.

Если $\dim V = n$, $\dim V' = m$, то $\Lb\hr{V, V'} \cong \Mat_{m, n}{\hr{\K}}$.
\begin{stm}
	$\Hom{\hr{\rho, \rho'}}$ --- подпространство в $\Lb\hr{V, V'}$.
\end{stm}
\begin{proof}
	Пусть $\phi_1, \phi_2 \in \Hom_G{\hr{\rho, \rho'}}$.
	\begin{gather*}
		\hr{\phi_1 + \phi_2}(gv) = \phi_1(gv) + \phi_2(gv) =
		g\phi_1(v) + g\phi_2(v) = g \hr{\phi_1(v) + \phi_2(v)} =
		g \hr{\phi_1 + \phi_2} (v) \\
		\hr{\lambda \phi_1}(gv) = g \hr{\lambda \phi_1} (v) \; \text{--- аналогичная проверка}
	\end{gather*}
\end{proof}

Рассмотрим $V' = V$.
$\Hom_G{\hr{V, V}} \subseq \Lb\hr{V}$ --- пространство лнейных операторов.
Пространство линейных операторов --- алгебра $\cong \Mat_n{\K}$.
\begin{stm}
	$\Hom_G{\hr{V, V'}}$ --- подалгебра в $\Lb\hr{V}$.
\end{stm}
\begin{comm}
	Композиция представлений --- представление.
\end{comm}
\begin{df}
	Эндоморфизм --- гомоморфизм на себя.

	Автоморфизм --- биективный эндоморфизм.
\end{df}
\begin{denote}
	$\End_G\hr{V}$ --- алгебра эндоморфизмов представлений в $V$
\end{denote}
\begin{lemma}[Шур]
	\begin{points}{-3}
		\item $\hr{G, V, \rho}$, $\hr{G, V', \rho'}$ --- неприводимые представления.
			Тогда $\fa \phi \cln \rho \to \rho'$ --- либо нулевой, либо биекция.
		\item $\End_G\hr{V}$ --- алгебра с делением
		\item $\K = \Cx$ $\Ra$ $\fa \phi \in \End_G\hr{V},
			\rho \; \text{неприводимое} \phantom{\cln} \phi = \lambda \epsilon, \lambda \in \Cx$
	\end{points}
\end{lemma}
\begin{proof}
	\begin{points}{-3}
		\item $\Ker\phi \subseq V$, $\Im \phi \subseq V'$ --- инвариантные подпространства.
			Но \bt\ $V$ и $V'$ --- неприводимые, то нет нетривиальных подпредставлений.
			$\Im\phi=0 \Ra \phi = 0$;
			$\Im \phi = V' \Ra \Ker \phi \ne V \Ra \Ker\phi \ne 0$ $\Ra$ $\phi$ --- биективено.
		\item Простое следствие пункта \pt{1}.
		\item Докажем двумя способами:
			\begin{nums}{-3}
				\item Пусть $\K = \Cx$.
					Тогда $\End_G\hr{V}$ --- $\Cx$\h алгебра с делением,
					$\dim_{\Cx}\End_G\hr{V} < \infty$ (\bt\ подалгебра в алгебре матриц) $\Ra$
					$\End_G\hr{V} = \Cx$.
				\item $\phi \cln V \to V$ --- эндоморфизм $\Ra$
					линейный оператор над $\Cx$ $\Ra$
					обладает хотя бы одним собственным вектором:
					$$
						\Ex  x \in V, x \ne 0, \lambda \in \Cx \cln 
						\phi(x) = \lambda x \Ra \hr{\phi - \lambda \epsilon}(x) = 0.
					$$
					Но любой эндоморфизм либо нулевой, либо биективен.
					Значит имеет тривиальное ядро $\Ra$
					$\phi - \lambda \epsilon = 0 \Ra \phi = \lambda \epsilon$.
			\end{nums}
	\end{points}
\end{proof}

\subsection{Гомоморфизмы прямой суммы представлений}
Пусть $\rho = \rho_1 \sop \rho_s$, $V = V_1 \sop V_s$, $\rho_i = \hr{G, V_i}$,
$\rho' = \hr{G, V'}$ --- любое.

Рассмотрим $\Hom_G\hr{V_1 \sop V_s, V'}$.
Гомоморфизм прямой суммы определяет гомоморфизм каждого слагаемого
$\phi \in \Hom_G\hr{V_1 \sop V_s, V'}$, $\phi_i \cln V_i \to V'$,
$$
	\phi(v) = \phi(v_1 \sco v_s) = \sum_{i = 1}^s{\phi_i(v_i)},
$$
$\fA v_i \in V_i \phantom{\cln} \phi_i(v_i) = \phi(v_i)$.

Но если $\fa i$ задано $\phi_i \cln V_i \to V'$, то
$$
	\phi(v) \deq \phi\hr{v_1 \spl v_s}
	= \sum_{i = 1}^s \phi_i(v_i)
$$
$\Ra$ $\phi \in \Hom\hr{\bigoplus V_i, V'}$ $\Ra$
$\Hom_G\hr{V_1 \sop V_s, V'} \cong \Hom_G\hr{V_1, V'} \sop \Hom_G\hr{V_s, V'}$.

Применим это к гомомоморфизму прямого произведения.
\begin{imp}[из Л.Шура]
	Пусть $V$, $V'$ --- неприводимые, тогда
	$$
		\dim \Hom_G\hr{V, V'} = \case{
			0, & V \ncong V' \\
			\dim & V \cong V'
			}
	$$
\end{imp}

$\rho\hr{G, V}$ --- вполне приводимо.
$$
	V = V_1 \sop V_s = V_1 \sop V_k \oplus V_{k + 1} \sop V_s,
$$
$V_i$, $\rho' = \hr{G, V'}$ --- неприводимы, $V_i \cong V'$, $1 \le i \le k$,
$V_i \ncong V'$, $k + 1 \le i \le s$.
$k$ --- кратность вхождения $V'$ в данное разложение вполне приводимого.

\begin{stm}
	$
		\dim_{\K}\Hom_G\hr{V, V'} = k \dim_{\K}\End_G\hr{V'}
	$
\end{stm}
\begin{proof}
	Из предыдущего
	\begin{gather*}
		\Hom_G\hr{V, V'} = \Hom_G\hr{V_1 \sop V_s, V'} = \\
		= \ub{\Hom_G\hr{V_1, V'}}_{\cong\End_G\hr{V'}} \sop \ub{\Hom_G\hr{V_k, V'}}_{\cong\End_G\hr{V'}} \oplus
		\ub{\Hom_G\hr{V_{k + 1}, V'}}_{=0} \sop \ub{\Hom_G\hr{V_{k + 1}, V'}}_{=0}
	\end{gather*}
	$\Ra$ $\dim \Hom_G \hr{V, V'} = k \End_G\hr{V'}$
\end{proof}
\begin{theorem}
	Кратность вхождения данного неприводимого представления
	в разложение вполне приводимого представления в прямую
	сумму не приводимых не зависит от выбора этого разложения
\end{theorem}
\begin{proof}
	$$
		k = \frac{\dim\Hom_G\hr{V, V'}}{\dim\End_G\hr{V'}}
	$$
	Легко видеть, что правая часть не зависит от выбора разложения, значит и левая.
\end{proof}
\begin{df}
	\textit{Кратностью} $\nu(V', V)$ неприводимого $V'$ в вполне приводимом $V$ называется
	кратность вхождения $V'$ в любое разложение $V$ в прямую сумму неприводимых.
\end{df}
\begin{imp}
	$\K = \Cx$, $\End_G\hr{V} = \Cx \Ra \nu(V', V) = \dim\Hom_G\hr{V, V'}$.
\end{imp}


\subsection{Ортоганальные и унитарные представления}
$\K = \R, \Cx$, $V$ --- линейное пространство, $\dim V < \infty$.
Зададаим на $V$ евклидово (в случае $\K = \Cx$, эрмитово) скалярное произведение.
\begin{df}
	V --- евклидово пространство.
	Тогда $\rho = \hr{G, V}$ называется \textit{ортоганальным}, если
	$\fA g \in G \rho(g)$ --- ортоганален.

	V --- эрмитово пространство.
	Тогда $\rho = \hr{G, V}$ называется \textit{унитарным}, если
	$\fA g \in G \rho(g)$ --- унитарен.

	Матричное представление называется ортоганальным (унитарным),
	если все $A_g$ --- ортоганальны (унитарны).
\end{df}

Если зададим ортогональные (унитарные) матрицы и ортонормированный (унитарный) базис,
то получим отогональное (унитарное) представление и наоборот.

\begin{theorem}
	Любое $\R$\h представление конечной группы изоморфно ортогональному,
	а над $\Cx$ --- унитраному.
\end{theorem}
\begin{proof}
Докажем для $\Cx$ (для $\R$ аналогично).
Достаточно доказать, что любое матричное представление
эквивалентно унитарному матричному.

Пусть $\rho = \hr{G, V}$ дано.
На $V$ есть эрмитово скалярное произведение,
относительно которого $\rho$ --- унитарно, \ie\
все $\rho(g)$ --- унитарные операторы.

Пусть $V = \ha{e_1 \sco e_n}$ --- базис,
$F(x, y) = \sum_{i = 1}^n \ol{x_i}y_i$ --- скалярное произведение.
Введём новое:
$$
	(x, y) \deq \sum_{g\in G} F(gx, gy)
$$
Покажем, что
\begin{points}{-3}
	\item получилось эрмитово скалярное произведение:
		$\fA g$ $F(gx, gy)$ --- положительно определённая полуторолинейная форма.
	\item относительно $(x, y)$ все операторы $\rho(g)$ унитарны, \ie\
		$\fA h \in G$ $\rho(h)$ --- унитарный. Но
		$$
			(hx, hy) = \sum_{g \in G} F(ghx, ghy) \quad \text{так же сумма по всей группе}
		$$
		$\Ra$ $(hx, hy) = (x, y)$ $\Ra$ $h$ --- унитарен.
\end{points}
\end{proof}
\begin{imp}
	Любое представление конечной группы над $\R$ или $\Cx$ --- вполне приводимы.
\end{imp}
\begin{proof}
	Индукция по размерности представления:
	\begin{points}{-3}
		\item Любое одномерное представление всегда неприводимо.
		\item Пусть $\dim V = n$ и для меньших размерностей доказано.
			Это представление изоморфно унитарному $\Ra$
			можем считать, что представление --- унитарно.
			Если неприводимо, то доказывать нечего.
			Если же существует инвариантное подпространство $0 \ne U \subsneq V$,
			то $U$, $U^\perp$ --- инвариантны относительно унитарных $\rho(g)$.

			$V = U \oplus U^\perp$, $\dim U, \, \dim U^\perp < n$,
			тогда по индуктивному предположению $U$ и $U^\perp$ --- прямая сумма неприводимых
			$\Ra$ $V$ разлагается в прямую сумму неприводимых.
	\end{points}
\end{proof}


\subsection{Критерий полной приводимости линейного представления над произвольным полем}
\begin{df}
	Представление $\rho$ на $V$ обладает свойтсвом \textit{отщепимости},
	если $\fA U' \subs V$ --- инвариантного $\Ex U'' \subs V \cln V$ --- инвариантное
	и $V = U' \oplus U''$
\end{df}
\begin{stm}
	Следующие свойства эквиваленты:
	\begin{points}{-3}
		\item $(G, V)$ --- вполне приводимо
		\item $(G, V)$ обладает свойством отщепимости
		\item $\fA U \subs V$ инвариантное $\Ex \phi \in \End_G\hr{V}$ --- проекция на $U$ 
	\end{points}
\end{stm}
\begin{proof}
	Эквивалентность \pt{2} и \pt{3} известна из линейной алгебры.

	Проекция $\phi$ --- гомоморфизм представлений $\Ra$
	$U'' = \Ker \phi$ --- инвариантно.

	Если $U''$ --- инвариантно, то проекция $\phi$ на $U'$ паралельно $U''$ --- гомоморфизм представлений:
	$\fA v = v' + v''$, $v' \in U'$, $v'' \in U''$ $\phi(v) = v'$.

	Докажем эквивалентность \pt{1} и \pt{2}.

	\pt{1} $\Ra$ \pt{2}:
	Пусть $U' \subs V$ --- инвариантное,
	$V = \ub{V_1 \sop V_s}_{\text{неприводимые}}$.

	Возьмём в качестве $U''$ \textbf{максимальную} сумму
	$U'' = V_{i_1} \sop V_{i_k}$, \sth\ $U'' \cap U' = 0$.

	\Bt\ пересечение нулевое, то $U' \oplus U''$ --- прямая сумма.
	Покажем, что $V = U' \oplus U''$.

	Пусть $U' \oplus U'' \subsneq V$ $\Ra$ $\Ex V_j \cln V_j \not \subs U' \oplus U''$
	Но $V_j$ --- неприводимо $\Ra$ $\hr{U' \oplus U''} \cap V_j = 0$ $\Ra$
	$U' \oplus U'' \oplus V_j$ --- прямая сумма, что противоречит максимальности $U''$ $\Ra$
	$U' \oplus U'' = V$.

	\pt{2} $\Ra$ \pt{1}:
	\begin{lemma}
		Если $V$ обладает свойством отщепимости,
		то любое его подредставление обладает этим свойством.
	\end{lemma}
	\begin{proof}
		$U \subs V$, $U' \subs U$ --- инвариантные,
		$V = U' \oplus U''$ --- инвариантное, тогда $U = U' \oplus \hr{U'' \cap U}$
		$$
			U' \cap \hr{U'' \cap U} = 0\text{, \bt\ }U' \cap U'' = 0
		$$
		$$
			\fA u \in U \phantom{\cln} u = v' + v'', \; v' \in U', \: v'' \in U'' \;
			v'' = u - v' \in U \Ra v'' \in U \cap U''
		$$
	\end{proof}
	Индукция по размерности $\dim V = n$
	\begin{points}{-3}
		\item $n = 1$ очевидно.
		\item если $V$ --- неприводимо, то доказывать нечего.
			Пусть $\Ex 0 \ne U' \subsneq V \Ra V = U' \oplus U''$ ---
			инвариантные и имеют меньшую размерность.
			По лемме они обладают свойтсвом отщепимости $\Ra$
			к ним применимо индуктивное предположение $\Ra$
			$V$ --- вполне приводимо.
	\end{points}
\end{proof}
\begin{imp}
	Подпредставление и факторпредставление вполне приводимого --- вполне приводимы.
\end{imp}
\begin{proof}
	% TODO: good reference
	Для подпредставления свойство наследуется по Лемме (1.13).

	Рассмотрим $V/U$.
	$V = U \oplus U' \Ra V/U \cong U'$, но
	$U'$ --- подпредставление $\Ra$ вполне приводимо.
\end{proof}
\begin{theorem}[Машке]
	Пусть $G$ --- конечная группа и поле $\K$.
	$\K$\h представление $G$ вполне приводимо $\Lra$ $\Char \K \ndivs \hm{G}$
\end{theorem}
\begin{df}
	Представление \textit{модулярно}, если $\Char \K \divs \hm{G}$.
\end{df}
\begin{proof}
% TODO: good reference
$\Lar$: Докажем с помощью условия (3) из критерия.

Из линейной алгебры:
$\Ex \psi \cln V \to V$, $\psi$ --- проекция на $U$.
Хотим получить проекцию $\phi$, \sth\
$\phi(hv) = h \phi(v) \fA v\in V, \: h \in G$.
Пусть $\hm{G} = n$, $0 \ne n \cdot 1 \in \K \Ra \frac 1 n = {(n \cdot 1)}^{-1}$.
Определим $\phi$:
$$
	\phi(v) = \frac 1 n \sum_{g \in G} g^{-1} \psi(gv)
$$
Докажем, что $\phi$ --- эндоморфизм.
$$
	\phi(hv) = \frac 1 n \sum_g g^{-1} \psi(ghv) = 
	\frac 1 n \sum_g h {(gh)}^{-1} \psi\hr{(gh)v} = 
	h \cdot \frac 1 n \sum_{gh = x \in G} x^{-1} \psi(xv) = h \phi(v)
$$
Проверим, что $\phi$ --- проекция на $U$.
\begin{gather*}
	\psi(gv) \in U \Ra \phi(v) = \frac 1 n \sum g^{-1} \psi(gv) \Ra \Im \phi \subseq U \\
	v \in U \Ra gv \in U \text{(инвариантность)}
	\Ra \psi(gv) = gv \Ra g^{-1}\psi(gv) = gv \Ra
	\psi(v) = \frac 1 n \cdot n \cdot v = v
\end{gather*}
\end{proof}

\subsection{Продолжение линейного действия группы на пространстве её представления вдоль действия её групповой алгебры}
Пусть $G$ --- группа, $\K$ --- поле, $\hm{G} = n$.
Рассмотрим $n$\h мерное векторное пространство,
отождествив элементы $G$ с базисом.
$\K G = \hc{\sum\limits_{g\in G} a_g g}$, $a_g \in \K$,
$$
	\hr{\sum_{g \in G} a_g g}\hr{\sum_{g \in G} b_g g} = 
	\sum_{g \in G}\hr{\sum_{h, k \cln hk = g} a_h b_k} g
$$

Пусть имеем представление $V$ над $\K$ $\rho = (G, V)$ $\Ra$
задано $gv$, $g \in G$, $v \in V$, $\rho \cln G \to \GL(V)$ --- невырожденные линейные операторы.
Но $G \subs \K G$ $\Ra$
$\fA \tau = \sum a_g g, \; g \in G$,
$$
	\tau v = \hr{\sum_g a_g g}(v) \deq \sum_g a_g (gv)
$$
Свойства этой операции:
\begin{points}{-3}
	\item $\tau (v_1 + v_2) = \tau v_1 + \tau v_2 \qquad
		\fA \tau \in \K G, \fA v_1, v_2 \in V$
	\item $\tau (\lambda v) = \lambda (\tau v) \qquad \fA \lambda \in \K$
	\item $\hr{\tau_1 + \tau_2} v = \tau_1 v + \tau_2 v$
	\item $\hr{\tau_1 \tau_2} v = \tau_1 \hr{\tau_2 v}$
	\item $\hr{\lambda e} v = \lambda v \qquad e$ --- единица в $\K G$, \ie\ в $G$, $\lambda \in \K$
\end{points}
Свойства \pt{1}, \pt{2} задают линейный оператор.
Раньше имели $\phi(g) v = gv$. А теперь $\phi \cln \K G \to \Lb(V)$ --- алгебра всех линейных операторв.

Свойства \pt{3}, \pt{4} и \pt{5} задают гомоморфизм алгебр
(линейное представление задаёт гомоорфизм алгебр $\K G \to \Lb(V)$).

Заметим, что имеют место свойства:
\begin{items}{-3}
	\item $U \subs V$ --- инвариантно относительно $G$ $\Ra$
		$\tau v \in U$, $\fA v \in U, \tau \in \K G$.
	\item $\phi \cln V \to V$ --- гомомморфизм представлений $\Ra$
		$\phi (\tau v) = \tau \phi(v)$, $\fa v \in V, \tau \in \K G$.
\end{items}

\begin{df}
	$\K$\h представление группы $G$ в $\K G$ и с действием,
	задаваемым умножением в групповой алгебре:
	$$
		g \hr{\sum a_h h} = \sum a_g g h
	$$
	называется \textit{регулярным}.
\end{df}


\subsection{Размерность пространства гомоморфизмов регулярного представления}
Пусть $\rho = (G, V)$ --- любое представление.
\begin{stm}
	$
		\dim \Hom_G \hr{\K G, V} = \dim_{\K} V
	$
\end{stm}
\begin{proof}
	$\fA \phi \cln \K G \to V$ задаётся значением $\phi(e) \in V$, \bt\
	$\phi(\tau) = \phi(\tau e) = \tau \phi(e)$.

	В обратную сторону: задано $\phi(e)$ $\Ra$
	$\fA \tau \in \K G$ $\phi(\tau) = \phi(\tau e) \deq \tau \phi(e)$.

	Покажем, что эта биекция $V \lra \Hom_G \hr{\K G, V}$ --- гомоморфизм векторных пространств.

	Пусть $v_0 = \phi(e)$, $v_o \lra \phi \in \Hom_G\hr{\K G, V}$.
	$\phi(\tau) = \tau v_0$. Возьмём $v'_0$ и $v''_0$, $v_0 = v'_0 + v''_0$,
	$\phi'(\tau) = \tau v'_0$, $\phi''(\tau) = \tau v''_0$.
	\begin{gather*}
		\phi(\tau) = \tau \hr{v'_0 + v''_0} = \tau v'_0 + \tau v''_0 =
		\phi'(\tau) + \phi''(\tau) = \hr{\phi' + \phi''}(\tau) \\
		\phi_{\lambda v'_0}(\tau) = \tau \hr{\lambda v'_0} =
		\lambda \phi'(\tau) = \hr{\lambda \phi'}(\tau) 
	\end{gather*}
	Значит имеем изоморфизм векторных пространств $\Ra$
	$\dim \Hom_G \hr{\K G, V} = \dim_{\K} V$.
\end{proof}


\subsection{Кратность вхождения неприводимого представления в немодулярном случае}
В немодулрном случае любое представление $G$ вполне приводимо $\Ra$
регулярное представление вполне приводимо $\Ra$
можно говорить о кратности вхождения неприводимого представления в регулярное.
$\rho = (G, V)$ --- неприводимое, $k = \nu(V, \K G)$.
% TODO: ref
Применим формулу:
$$
	k = \frac{\dim_{\K} \Hom_{G} \hr{\K G, V}}{\dim_{\K}\End_{G}\hr{V}}
	= \frac{\dim_{\K} V}{\dim_{\K}\End_{G}\hr{V}}
$$
$\Ra \fA V$ --- неприводимого, кратность его вхождения в $\K G$ ненулевая $\Ra$
любое неприврдимое встречается в разложении регулярного.

\Bt\ слагаемых в разложении конечное число,
то с точностью до изоморфизма имеется конечное число представлений группы $G$.

Если $\K = \Cx$, то $\dim_{\Cx} \End_{G} \hr{V} = 1$.
\begin{stm}
	Кратность вхождения неприводимого представления в регулярное над $\Cx$
	равно размерности пространства представления.
\end{stm}
\begin{imp}
	Сумма квадратов размерностей неприводимых представлений
	конечной группы $G$ над $\Cx$ равна $\hm{G}$.
\end{imp}
\begin{proof}
	$(G, V_1) \sco (G, V_s)$ --- список всех неприводимых представлений (с точночтью до изоморфизма)
	группы $G$ над $\Cx$, $\dim_{\K} V_i = k_i$,
	$$
		\K G = \ub{V_{1,1} \sop V_{1,k_1}}_{V_{1, j} \cong V_1} \sop
		\ub{V_{s,1} \sop V_{s,k_s}}_{V_{s, j} \cong V_s}
	$$
	$\Ra$ число слагаемых в каждой группе равно размерности представления
	$$
		\hm{G} = \dim_{\K} \K G = \sum_{i = 1}^s k_i^2
	$$
\end{proof}


\subsection{Разложение немодулярной групповой алгебры конечной группы в прямую сумму простых алгебр}
$\Char \K \ndivs \hm{G}$.
Тогда $(G, V_1) \sco (G, V_s)$ --- все неприводимые,
$k_i$ --- кратность вхождения $V_i$ в $\K G$
$$
	k_i = \frac{\dim_{\K}\Hom_G\hr{\K G, V_i}}{\dim_{\K}\End_G(V_i)}
$$
\begin{problem}
	Показать, что если $\Char K \divs \hm{G}$,
	то регулярное представление не является вполне приводимым.
\end{problem}
\begin{hint}
	Показать, что имеется подпредставление и на него нет проекции.
\end{hint}
\begin{equation}
	\K G = \obt{\ub{V_{1,1} \sop V_{1,k_1}}_{V_{1, j} \cong V_1}}^{R_1} \sop
	\obt{\ub{V_{s,1} \sop V_{s,k_s}}_{V_{s, j} \cong V_s}}^{R_s}
\end{equation}
Подпредставление в регулярном представлении --- подпространство,
инвариантное относительно умножения слева на элементы из $G$ $\Ra$
на любые элементы из алгебры
(\ie\ это левые идеалы в $\K G$).

$I \subs \K G$, $I$ --- неприводим $\Lra$
$I \neq 0$ и нет строго меньших ненулевых левых идеалов (минимальный левый идеал).
\begin{theorem}
	Немодулярная групповая алгебра является прямой суммой простых алгебр.
\end{theorem}
\begin{proof}
	Имеем разложение
	$\K G = V_{1, 1} \sop V_{1, k_1} \sop V_{s, 1} \sop V_{s, k_s}$.
		% TODO: ref
	Докажем, что блоки ($R_i$ из (5.1))--- двустороние идеалы.
	Заметим, что блоки определены однозначно.
	Докажем, что первый (значит и любой) блок содержит неприводимоё подредстовление $I \cong V_1$.

	Пусть $\Ex \tau_0 \in I, \: \tau_0 \notin R_i$,
	$\tau_0 = \sum \tau_{i, j}$, $\tau_{i, j} \in V_{i, j}$,
	$\Ex \tau_{i, j} \ne 0 \: i \ne 1$.

	Рассмотрим проекцию на $V_{i, j}$: $\phi \cln I \to V_{i, j}$,
	$\phi(\tau) = \tau_{i, j}$, $\tau \in I$.
	Но \bt\ было разложение в прямую сумму,
	то получили гомоморфизм представлений.
	Этот гомоморфизм не нулевой, \bt\
	
$\phi(\tau_0) \ne 0$.
	% TODO: ref
	Значит по Л.Шура $V_{i, j} \cong V_1$, что противоречит $i \ne 1$
	$\Ra I \subseq R_1 \Ra$ блоки не зависят от разложения.

	Докажем, что $R_1$ --- двустороний идеал.
	Осталось показать, что правый.

	Пусть $\tau \in \K G$, $R_1 \tau \subseq R_1$
	\begin{points}{-3}
		\item $V_{1, j} \cdot \tau = 0$
		\item $V_{1, j} \cdot \tau \ne 0$.
			Рассмотрим отображение $\phi \cln V_{1, j} \to V_{1, j} \cdot \tau$:
			$x \in V_{1, j}, \: \phi(x) = x \tau$.
			Заметим, что $\phi$ --- сюръективен.
	\end{points}

	Покажем, что $\phi$ --- гомоморфизм представлений:
	$$
		\phi(gx) = (gx) \tau = g \phi(x)
	$$

	Но $V_{i, j}$ было неприводимо,
	однако $\Ker \phi \subsneq V_{i, j} \Ra \Ker \phi = 0
	\Ra \phi$  --- изоморфизм представлений.
	$V_{1, j} \cdot \tau \cong V_{1, j} \cong V_1 \Ra V_{1, j} \cdot \tau \subseq R_1 \Ra$
	$R_i$ --- двустороний идеалы.

	Докажем, что $R_i$ --- простые.
	$\K G = R_1 \sop R_s$, $e = e_1 \sop e_s$, $e_i \in R_i$ --- единица в $R_i$

	\Bt\ $R_i$ --- двустороние идеалы, то произведение элементов из разных подалгебр равно нулю.
	$$
		\tau \in R_i \Ra \tau = \tau e =
		\ub{\tau e_1}_{= 0} \sop \ub{\tau e_i}_{= \tau} \sop \ub{\tau e_s}_{= 0}
	$$
	Пусть $0 \ne J \subsneq R_1$ --- двустороний идеал.
	\begin{align*}
		R_1 & = J \oplus J' \\
		R_1 & = \ub{I_1 \sop I_k}_{J} \oplus I_{k + 1} \sop I_{k_1}, \quad k \ne k_1
	\end{align*}
	$\fA \tau \in J$  $\tau I_{k + 1} \subseq I_{k + 1}$, \bt\ $I_{k + 1}$ --- подпредставление,
	$\tau I_{k + 1} \subseq J$, \bt\ $\tau \in J$, $J$ --- двустороний идеал.
	$J \cap I_{k + 1} = 0 \Ra \tau I_{k + 1} = 0$.
	$I_{k + 1} \cong V_1 \cong V_{1, j}$, $j = 1 \sco k_1$.
	Оператор действует одинаковым образом на изоморфных представлениях $\Ra$
	$\tau V_{1, j} = 0 \fA j \Ra \tau R_1 = 0 \Ra \tau = \tau e_1 = 0$. Противоречие.
\end{proof}
Рассмотрим $\K = \Cx$, $\Cx G = R_1 \sop R_s$.
\begin{theorem}
	Групповая алгебра конечной группы над $\Cx$ разлагется в прямую сумму полных матричных алгебр.
\end{theorem}
\begin{proof}
	Покажем, что $R_i$ изморфен полной матричной алгебре.
	$$
		\Cx G = V_{1, 1} \sop V_{1, r_1} \sop V_{s, 1} \sop V_{s, r_s}
	$$
	$r_i = \dim V_i$, \bt\ над $\Cx$ кратность вхождения в регулярное совпадает с размерностью.
	$\dim R_i = r_i^2$, покажем, что $R_1 \cong \Mat_{r_1}(\Cx)$
	% TODO:make it amasing, 60
\end{proof}
\begin{theorem}
	Число неприводимых $\Cx$\h представлений конечной группы $G$,
	с точностью до изоморфизма, равно числу её классов сопряжянных элементов.
\end{theorem}
\begin{proof}
	$$
		\Cx G = \Mat_{r_1}(\Cx) \sop \Mat_{r_s}(\Cx),\quad\text{$s$ --- число неприводимых представлений.}
	$$
	Посчитаем размерность центра групповой алгебры.
	$$
		Z(\Cx G) = Z\hr{\Mat_{r_1}(\Cx)} \sop Z\hr{\Mat_{r_s}(\Cx)}
	$$
	Но центры --- скалярные матрицы $\Ra$
	все слагаемые имеют размерность один $\Ra$
	$\dim Z(\Cx G) = s$.
	С другой стороны, $a \in \Cx G$, $a = \sum a_g g$,
	$a \in Z(\Cx G) \Lra ha = ah \fA h \in G$.
	\begin{gather*}
		h a h^{-1} = a \fA h \in G \\
		a = \sum_g a_g g \Ra h a h^{-1} = \sum_{g \in G} a_g h g h^{-1} \Ra \\
		a_g = a_{h g h^{-1}} \Lra a \in Z(\Cx G)
	\end{gather*}
	$t$ --- число классов сопряженности в $G$,
	$g_1 \sco g_t$ --- представители классов, $\Cl(g_i)$ --- класс сопряженности с $g_i$.
	$$
		a \in Z(\Cx G) \Lra a = \sum_{g_i} a_{g_i} \sum_{g \in \Cl(g_i)} g
	$$
	Значит, базис $Z(\Cx G)$ есть $\hc{\sum\limits_{g \in \Cl(g_i)} g | i = 1 \sco t}$ $\Ra$
	$$
		t = \dim Z(\Cx G) = s
	$$
\end{proof}
%TODO: 61 и далее
\subsection{Примеры построения всех неприводимых $\Cx$\h представлений}
\subsection{Неприводимые $\Cx$\h представления группы кватернионов}
{\Huge Тут могла бы быть Ваша реклама.}
\subsection{Примеры неприводимых $\Cx$\h представлений групп $A_n$ и $S_n$ ($n \ge 4$)}
\subsection{Все неприводимые $\Cx$\h представления групп $A_4$ и $S_4$}
\subsection{Характеры $\Cx$\h представлений конечных групп}
Пусть имеем группу $G$, $\hm{G} < \infty$.
Пусть имеем $\rho \cln G \to \GL(V)$,
$g \mapsto \rho(g) \in \GL(V)$,
$g \mapsto A_g$, $A_g$ --- матрица $\rho(g)$ в фиксированном базисе.
\begin{df}
	\textit{Характером} $\rho$ называется функция $\chi_{\rho} \cln G \to \Cx$,
	$\chi_{\rho}(g) = \tr \rho(g) = \tr A_g$.
	Заметим, что след совпадает с коэффициентом при $\lambda^{n - 1}$ в характеристическом многочлене,
	$\Ra$ определение не зависит от базиса.
\end{df}
Свойтва характера:
\begin{points}{-3}
	\item $\rho' \cong \rho \Ra \chi_{\rho'} = \chi_{\rho}$,
		\bt\ соответствующие матрицы сопряжены и следы совпадают.
	\item $g$ и $h$ сопряжены в $G$, то $\chi_{\rho}(g) = \chi_{\rho}(h)$,
		\bt\ соответствующие матрицы сопряжены $\Ra$
		$\chi_{\rho}$ постоянна на классах сопряжённых элементов.
	\item $\chi_{\rho}(g^{-1}) = \ol{\chi_{\rho}(g)}$

		$g^n = e$,
		$t^n - 1$ --- аннулирующий для $A_g$ $\Ra$
		все собственные значения матрицы $A_g$ --- корни из $1$.

		Характеристический многочлен, $m = \dim V$,
		$$
			f_{A_g}(\lambda) = \prod_{i = 1}^m \hr{\lambda - \lambda_i},
		$$
		$\chi_{\rho}(g) = \sum\limits_{i = 1}^m \lambda_i$, $A_{g^{-1}} = A_{g}^{-1}$.
		$\lambda_1 \sco \lambda_m$ --- корни с учётом кратности.
		\Bt\ $\lambda_i$ --- корни $n$\h й степени из $1$,
		то $\lambda_i^{-1} = \ol{\lambda_i}$ $\Ra$
		$$
			\chi_{\rho}(g^{-1}) = \sum_{i = 1}^m \lambda_i^{-1} =
			\sum_{i = 1}^m \ol{\lambda_i} = \ol{\chi_{\rho}(g)}.
		$$
	\item
		\begin{gather*}
			\rho = \bigoplus_{j = 1}^k \rho_j \Ra A_g =
			\rbmat{A_g^{(1)} && \\ & \ddots & \\ && A_g^{(k)}} \\
			\chi_{\rho} = \sum_{j = 1}^k \chi_{\rho_j}
		\end{gather*}
\end{points}


%TODO: should be amazing, 64
\subsection{Характеры как линейные функции на групповой алгебре}
\subsection{Характер регулярного представления группы $G$}
$\rho_{reg} \cln G \to \GL(\Cx G)$, $\hm{G} = n$,
$$
	\rho_{reg}(g)\hr{\sum_{h \in G} a_h h} = \sum_{h \in G} a_h gh
$$
\begin{align*}
	A_g & = \bcase{
			& E , & g &= e \\
			& \sum_{i, j \ne i} E_{i, j} , & g &\ne e \text{, \bt\ } \fA h gh \ne h
			} \\
	\chi_{reg}(g) & = \case{
				n , & g = e \\
				0 , & g \ne e
			}
\end{align*}

Пусть $a = \sum_{h \in G} a_h h$, $a g^{-1} = \sum_{h \in G} a_h h g^{-1}$
\begin{gather*}
	\chi_{reg}(a g^{-1}) = \sum_{h \in G} a_h \chi_{reg}(h g^{-1}) =
	n a_g, \quad \text{\bt\ все слагаемые равны, кроме случая, когда $h = g$} \\
	\Ra a_g = \frac 1 n \chi(a g^{-1})
	\Ra a = \frac 1 n \sum_{g \in G} \chi_{reg} (a g^{-1}) g
\end{gather*}
Последнее равенство --- формула разложения групповой алгебры по базису
в терминах характера регулярного представления.


\subsection{Характеры неприводимого $\Cx$\h представлений конечной группы}
% TODO: ping Vanya
