\section{Линейные представления (действия) групп}

\subsection{Определения}

Зафиксируем поле $\K$, над которым будем рассматривать векторное пространство
$V(+, \cdot, \cdot)$ (умножение на скаляр и на элементы $G$).

\begin{df}
	Задано \it{линейное действие}, если задано умножение элементов из $V$
	слева на элементы из $G$, $\fA g \in G, v \in V (g, v) \mapsto gv \in V,$
	\sth $\fA v, v_1, v_2 \in V \fA g, h \in G, \fA \la \in \K$
	\begin{points}{-3}
		\item $(gh)v = g(hv)$
		\item $ev = v$
		\item $g(v_1 + v_2) = gv_1 + gv_2$
		\item $g(\la v) = \la (gv)$
	\end{points}

	\it{Линейное представление} $\rh \colon G \to \GL(V)$, $\rh(g)(v) = gv$
	и обратно $gv = \rh(g)(v)$.
\end{df}
\begin{denote}
	$(G, V, \rh)$ эквивалентно записи $\rh = (G, V)$.
\end{denote}
\begin{df}
	Подпространство $U \subseteq V$ является {\it подпредставлением}, если
	оно инвариантно относительно действий элементов $G$, \ie
	$\fA u \in U \fA g \in G$ $gu \in U$. 
	То есть, грубо говоря, для каждого $g$ под его действием подпространство $U$ остается на месте.
\end{df}
\begin{df}
	Пусть у нас есть представление $V$, и его инвариантное подпредставление $U$. Тогда {\it факторпредставлением} $V/U$ назовем множество $\lbrace v + U | v \in V \rbrace$ с такой операцией: $g(v + U) = gv + U (\fA g \in G)$

	{\bf Проверка корректности.} Возьмем два разных предстваителя: $v_1 + U = v_2 + U$, т.е. $v_1 - v_2 \in U$. Тогда надо проверить, что $gv_1 + U = gv_2 + U$. Это правда, так как $g(v_1 - v_2) \in U$ (так как $U$ инвариантно) 
\end{df}

\begin{comm}
	На самом деле, все это дело очень похоже на теорию групп, которая была раньше. Довольно просто можно провести аналогию между подпредставлениями и нормальными подгруппами, и между факторпредставлениями и факторгруппами.
\end{comm}

\subsection{Прямая сумма представлений}

\begin{df} (Внутренняя прямая сумма.)
	Пусть заданы инвариантные подпространства $U_1, \ldots, U_s \subset V$, $V = U_1 \oplus \ldots \oplus U_s$ -- разложение в прямую сумму инвариантные подпространств. Тогда с каждым из этих подпространств свяжем подпредставление $(\rho_1 = (G, U_1), \rho_2 = (G, U_2), \ldots , \rho_s = (G, U_s))$ 
\end{df}

{\bf Внешняя прямая сумма.} Если есть пространства $V_1, \ldots, V_s$, тогда если взять $V = V_1 \oplus \ldots \oplus V_s$, и определить $g(v_1, \ldots, v_s) = (gv_1, \ldots, gv_s)$, то получим представление $\rho$, которое обозначим за $\rho_1 \oplus \ldots \oplus \rho_s$.

Пусть задан гомоморфизм $H \xrightarrow{f} G \xrightarrow{\rho} \GL(V)$. Тогда композицией $f$ и $\rho$ можно получить представление $H: hv = f(h)v$


\subsection{Гомоморфизмы представлений}

Пусть у нас есть два представлений одной группы $G$: $\rho_1 = (G, V_1)$ и $\rho_2 = (G, V_2)$. 
\begin{df}
	{\it Гомоморфизмом} представлений $\rho_1$ и $\rho_2$ назовем такое линейное отображение $\varphi: \rho_1 \rightarrow \rho_2$, что $\fA g \in G, v \in V$ $\varphi(gv) = g\varphi(v)$. То есть, в частности, и для любого $g$ должно быть верно, что $\rho_2(g) \circ \varphi = \varphi \circ \rho_1(g)$
\end{df}


\begin{df}
	{\it Изоморфизмом} двух представлений называется их гомоморфизм, который является биекцией.
\end{df}

\subsection{Матричные представления}
Возьмем представление $\rho = (G, V), \rho: G \rightarrow \GL(V)$. {\bf Всегда} будем считать, что $V$ -- конечномерное пространство, возьмем его базис $\lbrace e_1, \ldots , e_n \rbrace$, тогда $\GL(V) \cong \GL(n, \K)$ (группа невырожденных квадратных матриц $n \times n$ над полем $\K$) (здесь невырожденность нужна для того, чтобы для каждой матрицы всегда существовала обратная).

Теперь  каждому элементу $g \in G$ сопоставим матрицу $A_g$, а именно запишем оператор $\rho_g$ в базисе $\lbrace e_1, \ldots , e_n \rbrace$.
 \begin{df}
 	Таком образом, {\bf матричным представлением} группы $G$ назовем ее гомоморфизм в группу $\GL(n, \K)$.

 	Там выполняются следующие свойства:
 	\begin{points}{-3}
 		\item $A_{gh} = A_g A_h$
 		\item $A_e = E$
 		\item $A_{g^{-1}} = (A_g)^{-1}$
 	\end{points}
 \end{df}

Если у нас есть матричное представление 