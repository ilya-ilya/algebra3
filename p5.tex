\section{Линейные представления (действия) групп}

Зафиксируем поле $\K$, над которым будем рассматривать векторное пространство
$V(+, \cdot, \cdot)$ (умножение на скаляр и на элементы $G$).

\begin{df}
	Задано \it{линейное действие}, если задано умножение элементов из $V$
	слева на элементы из $G$, $\fA g \in G, v \in V (g, v) \mapsto gv \in V,$
	\sth $\fA v, v_1, v_2 \in V \fA g, h \in G, \fA \la \in \K$
	\begin{points}{-3}
		\item $(gh)v = g(hv)$
		\item $ev = v$
		\item $g(v_1 + v_2) = gv_1 + gv_2$
		\item $g(\la v) = \la (gv)$
	\end{points}

	\it{Линейное представление} $\rh \colon G \to \GL(V)$, $\rh(g)(v) = gv$
	и обратно $gv = \rh(g)(v)$.
\end{df}
\begin{denote}
	$(G, V, \rh)$ эквивалетно записи $\rh = (G, V)$.
\end{denote}
\begin{df}
	Подпространство $U \subseteq V$ является \it{подредставлением}, если
	оно инвариантно относительно действий элементов $G$, \ie
	$\fA u \in U \fA g \in G gu \in U$. 
\end{df}
